\begin{diary}{去宇宙深处摘几朵小花吧~!}{开篇小引}
看见@知乎 单纯猫 的一个陈年想法

大概两周之前我在 Youtube 上看到一个视频, 一位 Minecraft youtuber 介绍怎样玩 Minecraft 才能坚持玩下去而不感到厌烦, 他提到 document everything, 即记录一切, 有助于保持你玩下去的热情. 当时我觉得这帮老外真是吃饱了撑的: 玩个游戏至于吗, 不想玩就不玩呗. 但当我按照这个建议, 玩 Minecraft 的时候一边探索地图一边用 txt 写游记, 确实体会到了从未有过的乐趣. 后来我渐渐不记了, 因为每次暂停游戏打开 txt 有些麻烦. 但我的热情依旧未减.忽然想到, 学习不是一样的道理吗.

最近在忙科创(学不会),在看交换代数(报了代数几何的讨论班 有这个要求)

有点抑郁,沮丧的说.也许记下来今天看见了什么会好很多吧

附注:这个垃圾排版太丑了,回头改一改()
\end{diary}

\begin{diary}{CA还是很有趣哒}{2023.2.23}
对着commalg-2013复习了一下学过的CA,看到准素分解开了个头.
感觉正戏开始了w
很有意思的是局部化函子
升降链的本质区别是什么呢?
\end{diary}
