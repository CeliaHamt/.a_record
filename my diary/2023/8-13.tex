\begin{diary}{reflecting and planning}{2023.8.13}
    For th sake of convenience, I want to try to write my Math Diary in en. 绷不住了,\LaTeX 中英文切换是真的恶心,我英语也是真的烂(

    不过内容大于手段, 于是plan1就是想用markdown来代替tex的记录. 另外想把tex的版面设计地好看一点

    另一方面, 我希望更多地是记录每天所学中, 
    \begin{itemize}
        \item 直指某个理论核心的总结
        \item 有趣的应用(例子),或是经典的应用(例子)的总结
        \item 感觉有意思的My Ideal
    \end{itemize}

    之后学习的框架大概就是以数论为主线,大概就是曹阳老师的跟研要求:

    Part I. 基础内容
    \begin{itemize}
        \item 代数簇以及概型理论
        \item 代数数论(local fields, global fields, classical fields)
        \item 同调代数
        \item 表示论
    \end{itemize}

    Part II. 进阶内容
    \begin{itemize}
        \item Etale上同调
        \item 代数拓扑(Tammo tom Dieck)
        \item 复的内容
    \end{itemize}

    以上,对未来以及过去的自己土下座qaq


\end{diary}