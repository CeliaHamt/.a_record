\begin{diary}{集合论里的小花,维数的思考,庞加莱,三不}{2023.2.24}
    今天上了一天的数学课.王光辉老师讲的集合论很有意思,问的问题都很正.
    
    拓扑的老师也很有意思,里面对维数的思考刚好切合最近的欲求,基本的问题是圆和线段是否同胚,$\mathbb{R}^n$之间是否同胚,当然有人说可以去除第一维的子流形,从而用连通性作为拓扑不变量进行分析.

    我在想,要分辨两个东西,核心是用不同的拓扑不变量进行筛选比较.

    莫比乌斯环三分后的两个环嵌在一起应该怎么刻画呢.

    老师的三不:不怕错 不怕慢 不和别人比. 感觉受益匪浅,少焦虑了不少.

    李起峰老师讲了一下午的复分析(数学分析x),感觉他对Cauchy积分定理的议论挺有意思的.

    晚上听了一个看上去挺老的老师讲了3h倒向随机微分方程.感觉挺有意思的,思路打开hhh 感觉以后每学期都去学一门没怎么了解的课会比较好玩呢.
\end{diary}